\chapter{Erfolgsfaktoren eines CERT}

% **************************** Define Graphics Path **************************
\ifpdf
    \graphicspath{{Chapter3/Figs/Raster/}{Chapter3/Figs/PDF/}{Chapter3/Figs/}}
\else
    \graphicspath{{Chapter3/Figs/Vector/}{Chapter3/Figs/}}
\fi

\section{Aufbau - Kritische Erfolgsfaktoren}

\subsection{Einsatz von Ressourcen (Mitarbeiter, Zeit, Budget)}
Wie bei jedem Projekt muss sichergestellt werden, dass verfügbare Ressourcen korrekt und effektiv eingesetzt werden. Dabei sind sinnvolle Methodiken des Projektmanagementes anzuwenden. Mitarbeiter-Ressourcen, ihre verfügbare Zeit sowie auch das Budget sind hierbei die wichtigsten Faktoren. So muss z.B. sichergestellt werden, dass genügend Ressourcen vorhanden sind, jedoch ein Leerlauf möglichst verhindert wird. Hierbei gilt es jedoch darauf zu achten, dass der Einsatz von Ressourcen vernünftig ist. ``So wenig wie möglich`` ist hierbei der falsche Ansatz und führt zu mangelhafter Qualität.\\
\\
Im Vergleich zu anderen Erfolgsfaktoren sind diese mit gängigen Projektmanagement-Methoden messbar.

\subsection{Projektabschluss}
Analog zum Einsatz der Ressourcen ist bei Projekten auch der zeitlich vereinbarte Projektabschluss-Termin wichtig. Mit den angemessenen Projektplanungsmethoden kann die Wahrscheinlichkeit eines zu späten Abschlusses reduziert werden. Hierbei ist wichtig, dass mögliche Verspätungen schnellstmöglich kommuniziert werden, damit die Projektdauer entsprechend abgeschlossen werden kann. Verzögert sich der Abschluss des Projektes zu lange, besteht das Risiko, dass sich die Umgebung geändert hat und nicht mehr alle Bedürfnisse abgedeckt werden können. Eine zu grosse Fokussierung auf ``wir müssen das Projekt sofort abschliessen`` hindert jedoch die Umsetzung von qualitativ hochwertigen Ergebnissen.

\subsection{Unterstützung durch das Management}
Für beide oben genannten Erfolgsfaktoren ist es unabdingbar, dass das Management Unterstützung bietet. Es müssen Ressourcen geplant und Budget investiert werden. Schlussendlich sind dies Entscheidungen des Managements. Somit ist es wichtig, das Management frühzeitig und stetig in den Prozess des Aufbaus einzubinden und wo nötig zu konsultieren. In jedem Fall sollte das Management über den Projektstand regelmässig informiert werden. 

\subsection{Kundenglaubwürdigkeit}
Falls das Aufbau-Projekt von Kunden finanziert wird und nicht rein interner Natur ist, ist auch die Glaubwürdigkeit gegenüber der Kunden ein wichtiger Erfolgsfaktor. Der Kunde muss sicher sein, dass das aufzubauende Team in der Lage ist die gewünschten Dienstleistungen für die Kunden zu übernehmen. Hierbei ist z.B. wichtig, dass regelmässig über den Projektfortschritt kommuniziert wird, da auch im Betrieb des CSIRT schlussendlich Kommunikation einer der wichtigsten Bestandteile sein wird. Wird bereits während des Projektes nicht genügend kommuniziert, deutet dies auf eine Schwachstelle des Teams hin und es könnte davon ausgegangen werden, dass dies sich auch im Betrieb nicht ändern wird. Auch gegenüber Kunden, welche sich nicht finanziell am Projekt beteiligen, sollte regelmässig kommuniziert werden. Schlussendlich müssen die Kunden sicher sein, dass das Team ihre Anforderungen gewissenhaft, korrekt und zeitkritisch umsetzen kann.

\subsection{Bestehende Ressourcen / Hilfe von anderen CERTs}
Ein weiterer Erfolgsfaktor sind bestehende Ressourcen, bzw. Hilfe von anderen Teams. Soweit bestehende Ressourcen genutzt werden können, sollte auch darauf zugegriffen werden. Im CERT-Bereich sind viele Informationen öffentlich zugänglich. Diese sollten bereits beim Aufbau analysiert und eingebunden werden, da so das bestmögliche Ergebnis erreicht werden kann. \\
\\
Andere CERT können beim Aufbau wesentlich zur Qualität beitragen, da diese sich bereits im Betriebsstadium befinden und ggf. wichtige ``Lessions learned`` vermitteln können.

\section{Betrieb - Kritische Erfolgsfaktoren}

Beim Betrieb kommen fast die selben Kategorien von Erfolgsfaktoren zum Einsatz, da viele der Faktoren auch weiterhin gültig sind. 

\subsection{Einsatz von Ressourcen (Mitarbeiter, Zeit, Budget)}
Auch während des Betriebs muss sichergestellt werden, dass genügend Ressourcen vorhanden sind. Ein unterbesetztes CSIRT kann nicht die geforderten Leistungen erbringen und senkt somit die Kundenzufriedenheit. Wie in jeder anderen Organisation muss jedoch auch sichergestellt werden, dass die vorhandenen Ressourcen auch nach Projektabschluss vernünftig eingesetzt werden. 

\subsection{Unterstützung durch das Management}
Die Unterstützung des Managements ist weiterhin wichtig. Obwohl nach Projektabschluss ggf. ein Jahres-Budget vorhanden ist, muss wie bei jeder anderen Organisation auch, darauf geachtet werden, dass Änderungen der Managementstrategie in Einklang mit den Zielen des CSIRT gebracht werden. Ein kontinuierlicher Austausch zwischen Management und dem CSIRT-Team ist unabdingbar.

\subsection{Angebotene Dienstleistungen / Kundenzufriedenheit}
Erst während des Betriebs kann ermittelt werden, ob die angeboten Dienstleistungen den Kundenanforderungen entsprechen oder sich diese mittlerweile in eine andere Richtung bewegt haben. Die Kundenzufriendenheit betreffend Reaktionszeit, Qualität und Beratung kann z.B. mittels eines Fragebogens regelmässig überprüft werden. Zudem sollte regelmässig eine Analyse der Kunden (mit SWOT oder PEST) durchgeführt werden, um sicherzustellen, dass die angebotenen Dienstleistungen weiterhin die bestmögliche Qualität aufweisen. Dies ist einer der wichtigsten Faktoren für den Betrieb, da ohne Kunden (extern sowie auch intern) das Team nicht mehr benötigt wird und Dienstleistungen von anderen Anbietern angefordert werden.

\subsection{Dokumentation}
Um die Qualität und Zufriendeheit auf einem hohen Niveau zu halten, muss die Dokumentation immer wieder aktualisiert werden und allfällige ``Lessions learned`` angewendet werden. Nur so ist es möglich eine kontinuierlich hohe Qualität zu leisten und die Prozesse weiter zu verbessern. Die Dokumentation jedes Falles in einem ausreichenden Mass kann in Zukunft sehr wertvoll sein, auch wenn dies im ersten Augenblick nicht der Fall zu schein mag. Die Dokumentation hilft auch dabei neuen Mitarbeitern den Einstieg leichter zu machen und bereits vorhandenes Wissen zu vermitteln.

\subsection{Bestehende Ressourcen / Hilfe von anderen CERTs}
Während des Betriebs kann weiterhin auf die Mithilfe von anderen CERTs sowie auch auf den Zugriff von bestehenden Ressourcen gezählt werden. Je besser die Zusammenarbeit mit anderen Teams funktioniert, desto besser können die Kunden in ihren Anforderungen unterstützt und kompetent bedient werden.

